\section{Binding overview}\label{bindingoverview}

%Classes, functions and macros: \helpref{wxDebugContext}{wxdebugcontext}, \helpref{wxObject}{wxobject}, \helpref{wxLog}{wxlog},
%\rtfsp\helpref{Log functions}{logfunctions}, \helpref{Debug macros}{debugmacros}

\wxheading{Introduction}

A binding is a library of wrapper code that enables another language to use the wxWidgets C++ classes natively as if it were written
in that other language. I will concentrate on binding from C++ (the source language) to another object-oriented language (the target
language).

Inside the target language a separate class\footnote{The target language may not call the class by this name.} hierarchy will be
created, this hierarchy will mirror the C++ class hierarchy such that when a class in the target language is instantiated the C++
class will also be instantiated at the same time.

\subsection{Wrapper functions}\label{wrapperfunctions}

There is no standard for C++ name mangling and every compiler mangles names differently, so importing C++ classes into another language
can be tricky or just impossible. Not only that, even if the target language does support importing classes, different compilers generate
different names thus the target language's compiler would have to be psychic!

Due to this it is necessary to create a number of functions which flatten out to the objects into simple function calls with the object
instance (this) as the first parameter. This is what the compiler for an OO language generates anyway.

\subsection{Application initialisation}\label{applicationinitialisation}

This is one of the most complex parts of the binding process due to the way wxWidgets initialises itself. The target language will
probably handle command line arguments differently so using wxEntry is the best way forward. wxEntry calls wxApp::OnInit, this is
where the complications arise; constructing the app before wxWidgets has initialised itself will cause a crash, yet the app will want
to initialise itself, hence a cyclic initialisation sequence exists.

\subsection{Class wrapping}\label{classwrapping}

Certain classes need to be derived from inside the source language before they can be wrapped into the target language. This will be
due to pure virtual functions that have to be derived, e.g. Clone. Some of these virtuals will need to be wrapped so that the call can be
forwarded to the target language.

\begin{itemize}\itemsep=0pt
\item wxApp
\item wxEvent
\item wxEvtHandler
\item wxSizer
\item wxValidator
\item wxWindow
\end{itemize}

\subsection{Event handling}\label{eventhandling}

Event's are a very tricky part of the binding process as two different scenarios will show:

\begin{enumerate}\itemsep=0pt
\item Deriving a new event class from within the target language.
\item Forwarding C++ event instances to the target language.
\end{enumerate}

In both cases, both the C++ class instance and the target language's class instance need to be available. In the first case the event
thunker must get hold of the target language's instance and forward that onto the target language's event handling code. In the second
case the event thunker must generate a target language wrapper instance which will be destroyed as soon as possible and must not own the
source language's instance.

Another area where a binding will probably differ is the event tables. wxAda and wxPython do not use wxEventTable. In C++ the wxEventTable
is built up using the EVT_*() macros, this is not possible under wxAda and wxPython and they use the wxEvtHandler::Connect member function
instead.

Examples of how different bindings handle the thunking can be found here:

\begin{itemize}\itemsep=0pt
\item{\bf wxPython/src/helpers.cpp} wxPyCallback::EventThunker
\item{\bf wxEiffel/elj-2002/lib/ifs/c/ewxw/wrapper.cpp} ELJApp::HandleEvent
\item{\bf wxAda/src/c/wxbase_evthandler.cc} wxAdaCallbackObject::EventHandler
\end{itemize}

\subsection{Virtual functions}\label{virtualfunctions}

Any virtual function in a source language class needs to be forwarded onto the target language's class.

Source class virtual calls a target language virtual proxy, which in turn calls the target language's virtual. The target language's
virtual then needs to call the base virtual.

{\small
\begin{verbatim}
class wxTargetEvtHandler : public wxEvtHandler
	{
	public:
		...
		
		// If this function is called on an instance of any class derived from this class from within the wxWidgets library
		// the correct virtual function inside the target language will be called.
		virtual bool ProcessEvent(wxEvent& event)
			{
			// Call the target language virtual proxy, or if possible, call the virtual directly.
			}
		
		inline bool Base_ProcessEvent(wxEvent& event)
			{
			return wxEvtHandler::ProcessEvent(event);
			}
			
		...
	};

extern "C"
	{
	...

	// This function is made available so that the correct source language member function is called from the target language.
 	int wxEvtHandler_Base_ProcessEvent(wxTargetEvtHandler *Self, wxEvent *Event)
		{
		return Self->Base_ProcessEvent(*Event) == true ? 1 : 0;
		}
	
	...
	};
\end{verbatim}
}%

The virtual call in the binding language can then decide whether or not to call the base implementation from within itself.

In some bindings, e.g. wxPython, wxLua, etc. a test is made for the presence of a virtual function in the target language's instance and
if it exists call it, otherwise call the base class. In wxAda it is enough to inherit a new C++ class which forwards the virtual call onto
the Ada primitive sub-program (i.e. member function) as the correct primitive will be called and any Ada derived tagged type (i.e. class)
which overrides that particular sub-program will call the parent's sub-program if it needs to.

Examples of how different bindings handle virtual functions can be found here:

\begin{itemize}\itemsep=0pt
\item{\bf wxLua254/Library/wxLuaPrinting.cpp} wxLuaPrintout::HasPage
\item{\bf wxPython/src/_treectrl.i} wxPyTreeCtrl::OnCompareItems
\item{\bf wxAda/src/c/} 
\end{itemize}

